\documentclass[11pt, a4paper, tikz]{book}

\usepackage[spanish]{babel} %language
\usepackage[utf8]{inputenc} %UTF-8 encoding
\usepackage[margin=1.2in]{geometry} %smaller margin
\usepackage{amssymb} %\nexists \mathbb
\usepackage{amsmath}
\usepackage{amsthm}
\usepackage{graphicx}
\usepackage{polynom}
\usepackage{mathtools}
\usepackage{enumitem}

\usepackage{xlop}
\usepackage[dvipsnames]{xcolor}
\usepackage{tcolorbox}

\usepackage[
	%backend=bibtex,
	backend=biber,
	style=alphabetic,
	sorting=ynt,
]{biblatex}

\usepackage{listings}

\definecolor{pythonKeywordColor}{RGB}{100, 100, 255}
\definecolor{pythonStringColor}{RGB}{255, 100, 100}
\definecolor{pythonCommentColor}{RGB}{100, 175, 100}

\lstset{ 
	language=Python,         % Set the language to Python
	basicstyle=\ttfamily\scriptsize,     % Use a monospaced font
	keywordstyle=\color{pythonKeywordColor},% Python keywords will be blue
	stringstyle=\color{pythonStringColor},  % Strings will be red
	commentstyle=\color{pythonCommentColor},% Comments will be green
	showstringspaces=false,   % Don't show spaces in strings
	breaklines=true           % Break lines if they are too long
}

\definecolor{primaryEdgeColor}{RGB}{0, 0, 0}
\definecolor{secondaryEdgeColor}{RGB}{200, 200, 200}

\definecolor{formulationFrameColor}{RGB}{255,255,255}
\definecolor{formulationBackgroundColor}{RGB}{240,245,250}

\definecolor{draftFrameColor}{RGB}{255,255,255}
\definecolor{draftBackgroundColor}{RGB}{255,235,220}

\newtcolorbox{formulationBox}{
	colframe=formulationFrameColor,
	colback=formulationBackgroundColor
}

\newtcolorbox{draftBox}{
	colframe=draftFrameColor,
	colback=draftBackgroundColor
}

\newtheorem*{lemma}{Lema}
\renewenvironment{proof}{
	\textit{Demostración.}
}{

	$\qed$
}

\usepackage{tikz}   
\usepackage{pgfplots}

\pgfplotsset{compat=1.6}

\pgfplotsset{soldot/.style={color=red,only marks,mark=*}} \pgfplotsset{holdot/.style={color=red,fill=white,only marks,mark=*}}

\pgfplotsset{compat=newest} % used to declare my style command.  

\newcommand{\newpara}{
	\vskip 2mm
}

\newcommand{\myOver}[2]{
	\ensuremath{\overset{\kern2pt #1}{#2}}
}

\newcommand{\final}[1]{
	$\mathcal{F}($#1$)$
}

\renewcommand{\qed}{\hfill\blacksquare}

\newcommand{\Lim}[1]{\raisebox{0.5ex}{\scalebox{0.8}{$\displaystyle \lim_{#1}\;$}}}
\newcommand{\Inf}[1]{\raisebox{0.5ex}{\scalebox{0.8}{$\displaystyle \inf_{#1}\;$}}}
\newcommand{\Sup}[1]{\raisebox{0.5ex}{\scalebox{0.8}{$\displaystyle \sup_{#1}\;$}}}
\newcommand{\Sum}[2]{\displaystyle \sum_{#1}^{#2}}
\newcommand{\Int}[2]{\displaystyle \int_{#1}^{#2}}
\newcommand{\Bigcup}[2]{\displaystyle \bigcup_{#1}^{#2}}

\newcommand{\naturals}{
	\ensuremath{\mathbb{N}}
}
\newcommand{\integers}{
	\ensuremath{\mathbb{Z}}
}
\newcommand{\rationals}{
	\ensuremath{\mathbb{Q}}
}
\newcommand{\reals}{
	\ensuremath{\mathbb{R}}
}
\newcommand{\complexes}{
	\ensuremath{\mathbb{C}}
}

\newcommand{\cover}[1]{
	\ensuremath{\mathcal{#1}}
}

\newcommand{\sigmaAlgebra}[1]{
	\ensuremath{\mathcal{#1}}
}

\DeclareMathOperator{\trace}{tr}

\graphicspath{ {./media/} }
\usetikzlibrary{graphs,graphs.standard}
