\chapter{Preliminares sobre Álgebra Lineal}

\section{Ejercicio 1}

\begin{formulationBox}
	Sea $(K, +, \cdot)$ un cuerpo. Demostrar que
	\[K^n = \{(k_1,\dots,k_n)|k_i\in K, \forall i\in\{1, 2, \dots, n\}\}\]
	con la suma definida por: para cualesquier $(x_1, \dots, x_n), (y_1, \dots, y_n) \in K^n$
	\[(x_1, \dots, x_n) + (y_1, \dots, y_n) = (x_1+y_1, \dots, x_n+y_n)\]
	y la multiplicación por un escalar:
	\[\forall\lambda\in K, \forall(x_1, \dots, x_n)\in K^n, \lambda(x_1, \dots, x_n) = (\lambda x_1, \dots, \lambda x_n)\]
	es un $K$-espacio vectorial.
\end{formulationBox}

Primero, notemos que, como $(K, +, \cdot)$ es un cuerpo, también tenemos que $(K, +)$ es un grupo abeliano.

También, la multiplicación dada tiene dominio $K\times K^n$ y codominio $K^n$, un requisito para que $(K^n, +, \cdot)$ sea un $K$-espacio vectorial.

Que $(K^n, +)$ es un grupo abeliano es algo trivial teniendo en cuenta que $(K, +)$ también lo es, puesto que la suma de elementos de $K^n$ no es más que varias sumas de elementos de $K$, y compartirán las mismas propiedades.

Por otra parte, comprobamos las propiedades que la multiplicación por escalar debe satisfacer. En las próximas líneas, llamamos $v$ o $w$ a elementos cualesquiera de $K^n$, $v_i$ o $w_i$ a los elementos que forman dichas tuplas, $\lambda$ o $\lambda_i$ a elementos cualesquiera de $K$ y $1_K$ al neutro multiplicativo de $K$.
\begin{enumerate}[label=\alph*)]
	\item
	\begin{align*}
		1_K\cdot v &= 1_K\cdot(v_1,\dots,v_n)\\
		&= (1_K\cdot v_1,\dots,\cdot v_n)\\
		&= (v_1,\dots,v_n)\\
		&= v.
	\end{align*}
	\item
	\begin{align*}
		(\lambda_1 + \lambda_2)\cdot v &= (\lambda_1 + \lambda_2)\cdot(v_1,\dots,v_n)\\
		&= ((\lambda_1 + \lambda_2)v_1,\dots,(\lambda_1 + \lambda_2)v_n)\\
		&= (\lambda_1v_1 + \lambda_2v_1,\dots,\lambda_1v_2 + \lambda_2v_2)\\
		&= (\lambda_1v_1,\dots,\lambda_1v_2) + (\lambda_2v_1,\dots,\lambda_2v_2)\\
		&= \lambda_1(v_1,\dots,v_n) + \lambda_2(v_1,\dots,v_n)\\
		&= \lambda_1v + \lambda_2v.
	\end{align*}
	\item
	\begin{align*}
		\lambda\cdot(v + w) &= \lambda\cdot((v_1,\dots,v_n) + (w_1,\dots,w_n))\\
		&= \lambda\cdot(v_1+w_1,\dots,v_n+w_n)\\
		&= (\lambda(v_1+w_1),\dots,\lambda(v_n+w_n))\\
		&= (\lambda v_1+ \lambda w_1,\dots,\lambda v_n+\lambda w_n)\\
		&= (\lambda v_1, \dots, \lambda v_n) + (\lambda w_1, \dots, \lambda w_n)\\
		&= \lambda(v_1,\dots,v_n) + \lambda(w_1,\dots,w_n)\\
		&= \lambda v + \lambda w
	\end{align*}
	\item
	\begin{align*}
		(\lambda_1\lambda_2)\cdot v &= (\lambda_1\lambda_2)\cdot(v_1,\dots,v_n)\\
		&= ((\lambda_1\lambda_2)v_1,\dots,(\lambda_1\lambda_2)v_n)\\
		&= (\lambda_1(\lambda_2v_1),\dots,\lambda_1(\lambda_2v_n))\\
		&= \lambda_1(\lambda_2v_1,\dots,\lambda_2v_n)\\
		&= \lambda_1(\lambda_2(v_1,\dots,v_n))\\
		&= \lambda_1\cdot(\lambda_2\cdot v)
	\end{align*}
\end{enumerate}

Ya que se cumplen todas las propiedades necesarias, queda demostrado que $(K^n, +, \cdot)$ es un $K$-espacio vectorial.

$\qed$

\section{Ejercicio 2}

\begin{formulationBox}
	Estudiar si los siguientes conjuntos son $\mathbb{F}_q$-subespacios vectoriales de $\mathbb{F}_q^n$ y determinar su dimensión.
	\begin{enumerate}[label=\alph*)]
		\item $S_1=\{(a,a,\dots,a)|a\in\mathbb{F}_q\}\subseteq\mathbb{F}_q^n$.
		\item $S_2=\{(x_1,\dots,x_n)\in\mathbb{F}_q^n|x_n=\sum_{i=1}^{n-1}x_i\}$.
	\end{enumerate}
\end{formulationBox}

\begin{enumerate}[label=\alph*)]
	\item Para cualesquier $(x,\dots,x),(y,\dots,y)\in S_1$ y cualesquier escalares $\alpha,\beta\in\mathbb{F}_q$, se cumple
	\[\alpha(x,\dots,x) + \beta(y,\dots,y) = (\alpha x + \beta y,\dots,\alpha x + \beta y),\]
	que también es un vector de $S_1$ puesto que $\alpha x + \beta y \in\mathbb{F}_q$, luego $S_1$ es un $\mathbb{F}_q$-subespacio vectorial de $\mathbb{F}_q^n$.
	
	Para calcular la dimensión de $S_1$, ideamos primero una base $\mathcal{B}_1 = (1,\dots,1)$. Es una base ya que
	\begin{enumerate}[label=\arabic*)]
		\item Para todo $(x,\dots,x)\in S_1$ se escribe:
		\[(x,\dots,x) = x(1_{\mathbb{F}_q},\dots,1_{\mathbb{F}_q})\]
		con $x\in\mathbb{F}_q$, luego $\mathcal{B}_1$ es un sistema generador para $S_1$.
		\item El conjunto $\mathcal{B}_1$ es libre ya que
		\[(0_{\mathbb{F}_q},\dots,0_{\mathbb{F}_q}) = \alpha(1_{\mathbb{F}_q},\dots,1_{\mathbb{F}_q}) = (\alpha,\dots\alpha)\]
		implica $\alpha=0_{\mathbb{F}_q}$.
	\end{enumerate}
	El cardinal de esta base es $|\mathcal{B}_1| = 1$, puesto que tiene un único elemento. Por lo tanto,
	\[\dim_{\mathbb{F}_q}(S_1) = 1.\]
	
	\item Para cualesquier $(x_1,\dots,\sum_{i=1}^{n-1}x_i),(y_1,\dots,\sum_{i=1}^{n-1}y_i)\in S_2$ y cualesquier escalares $\alpha,\beta\in\mathbb{F}_q$, se cumple
	\[\alpha(x_1,\dots,\sum_{i=1}^{n-1}x_i) + \beta(y_1,\dots,\sum_{i=1}^{n-1}y_i) = (\alpha x_1 + \beta y_1,\dots,\alpha \sum_{i=1}^{n-1}x_i + \beta \sum_{i=1}^{n-1}y_i)\]
	que satisface
	\[\sum_{i=1}^{n-1}(\alpha x_i + \beta y_i) = \sum_{i=1}^{n-1}\alpha x_i + \sum_{i=1}^{n-1}\beta y_i = \alpha \sum_{i=1}^{n-1}x_i + \beta \sum_{i=1}^{n-1}y_i,\]
	luego $S_2$ es un $\mathbb{F}_q$-subespacio vectorial de $\mathbb{F}_q^n$.
	
	Para calcular la dimensión de $S_2$, ideamos primero una base
	\[\mathcal{B}_2 = \{(1_{\mathbb{F}_q},0_{\mathbb{F}_q},\dots,0_{\mathbb{F}_q},1_{\mathbb{F}_q}), (0_{\mathbb{F}_q},1_{\mathbb{F}_q},\dots,0_{\mathbb{F}_q},1_{\mathbb{F}_q}),\dots,(0_{\mathbb{F}_q},0_{\mathbb{F}_q},\dots,1_{\mathbb{F}_q},1_{\mathbb{F}_q})\}.\]
	Es una base ya que
	\begin{enumerate}[label=\arabic*)]
		\item Para todo $(x_1,\dots,\sum_{i=1}^{n-1}x_i)\in S_2$ se escribe:
		\[
			\begin{array}{r@{\ }l@{\ }l@{\ }l}
				(x_1,\dots,\sum_{i=1}^{n-1}x_i)&= \mathrlap{(x_1,x_2\dots,x_{n-1},x_1+\dots+x_n)}\\
				&= x_1(1_{\mathbb{F}_q},0_{\mathbb{F}_q},\dots,0_{\mathbb{F}_q},1_{\mathbb{F}_q}) &+& x_2(0_{\mathbb{F}_q},1_{\mathbb{F}_q},\dots,0_{\mathbb{F}_q},1_{\mathbb{F}_q})\\
				&	&+& \dots\\
				&	&+& x_{n-1}(0_{\mathbb{F}_q},0_{\mathbb{F}_q},\dots,1_{\mathbb{F}_q},1_{\mathbb{F}_q})
			\end{array}
		\]
		con $x_i\in\mathbb{F}_q\forall i\in\{1,\dots,n-1\}$, luego $\mathcal{B}_2$ es un sistema generador para $S_2$.
		
		\item El conjunto $\mathcal{B}_2$ es libre ya que
		\begin{align*}
			(0_{\mathbb{F}_q},\dots,0_{\mathbb{F}_q}) = \alpha_1(1_{\mathbb{F}_q},0_{\mathbb{F}_q},\dots,0_{\mathbb{F}_q},1_{\mathbb{F}_q}) &+ \alpha_2(0_{\mathbb{F}_q},1_{\mathbb{F}_q},\dots,0_{\mathbb{F}_q},1_{\mathbb{F}_q})\\
			&+ \dots\\
			&+ \alpha_{n-1}(0_{\mathbb{F}_q},0_{\mathbb{F}_q},\dots,1_{\mathbb{F}_q},1_{\mathbb{F}_q})
		\end{align*}
		implica
		\[\alpha_1=\alpha_2=\dots=\alpha_{n-1}=0.\]
	\end{enumerate}
	El cardinal de esta base es $|\mathcal{B}_2| = n-1$, puesto que tiene $n-1$ elementos. Por lo tanto,
	\[\dim_{\mathbb{F}_q}(S_2) = n-1.\]
\end{enumerate}

\section{Ejercicio 4}

\begin{formulationBox}
	Sea $S$ el subespacio vectorial de $\mathbb{F}_2^7$ definido por
	\[S=<1101000, 0110100, 0011010, 0001101>.\]
	\begin{enumerate}[label=\alph*)]
		\item Halla una base de $S$ y la dimensión de $S$.
		\item Estudia si la palabra $\textbf{x} = 1000110$ pertenece a $S$ y, en caso de que esté, calcula las coordenadas de $\textbf{x}$ en la base calculada en el apartado anterior.
	\end{enumerate}
\end{formulationBox}

\begin{enumerate}[label=\alph*)]
	\item La definición está dada por el sistema generador $\mathcal{B} = \{1101000, 0110100, 0011010, 0001101\}$. Este sistema generador es, a su vez, una base; podemos comprobar que todas sus palabras son libres ya que
	\begin{align*}
		0000000 &= \alpha(1101000) + \beta(0110100) + \gamma(0011010) + \delta(0001101)\\
		&= (\alpha)(\alpha+\beta)(\beta+\gamma)(\alpha+\gamma+\delta)(\beta+\delta)(\gamma)(\delta)
	\end{align*}
	implica
	\[\alpha=\beta=\gamma=\delta=0.\]
	La dimensión del subespacio será
	\[\dim_{\mathbb{F}_2^7}S = |\mathcal{B}| = 4.\]
	
	\item Tratemos de escribir $\textbf{x} = 1000110$ como combinación lineal de las palabras de la base:
	\begin{align*}
		1000110 &= \alpha(1101000) + \beta(0110100) + \gamma(0011010) + \delta(0001101)\\
		&= (\alpha)(\alpha+\beta)(\beta+\gamma)(\alpha+\gamma+\delta)(\beta+\delta)(\gamma)(\delta).
	\end{align*}
	De ahí, obtenemos las siguientes ecuaciones:
	\begin{align*}
		\alpha &= 1\\
		\alpha+\beta &= 0\\
		\beta+\gamma &= 0\\
		\alpha+\gamma+\delta &= 0\\
		\beta+\delta &= 1\\
		\gamma &= 1\\
		\delta &= 0.
	\end{align*}
	Vemos que
	\[\alpha=1, \beta=1, \gamma=1, \delta=0\]
	cumplen todas las ecuaciones, luego $\textbf{x}\in S$ y sus coordenadas son $(1,1,1,0)$.
\end{enumerate}

\section{Ejercicio 5}

\begin{formulationBox}
	Estudiar si el conjunto
	\begin{align*}
		S = \{&0000000, 0110100, 0011010, 0001101,\\
		&1000110, 1001011, 1011100, 0010111,\\
		&1010001, 1110010, 0111001, 1111111,\\
		&0101110, 1101000, 0100011, 1100101\}
	\end{align*}
	es un $\mathbb{F}_2$-subespacio vectorial de $\mathbb{F}_2^7$. En caso de que lo sea, calcula la dimensión de $S$.
\end{formulationBox}

Notemos que $|S| = 16 = 2^4$. Ya que $|\mathbb{F}_2| = \{0, 1\}$, la única forma de que una base generara exactamente $16$ palabras sería que esta tuviera cardinalidad $4$.

Tomamos cuatro palabras de $S$ que sean linealmente independientes:
\[\mathcal{B} = \{0110100, 0011010, 0001101, 1000110\}.\]

Verificamos que son linealmente independientes:
\begin{align*}
	0000000 &= \alpha(0110100) + \beta(0011010) + \gamma(0001101) + \delta(1000110)\\
	&= (\delta)(\alpha)(\alpha+\beta)(\beta+\gamma)(\alpha+\gamma+\delta)(\beta+\delta)(\gamma).
\end{align*}
implica
\[\alpha=\beta=\gamma=\delta=0.\]

Ahora tratamos de dar coordenadas a todos los elementos de $S$:
\[
	\footnotesize
	\begin{array}{r@{\ }l@{\quad}r@{\ }l@{\quad}r@{\ }l@{\quad}r@{\ }l}
		0000000&=(0,0,0,0) & 0110100&=(1,0,0,0) & 0011010&=(0,1,0,0) & 0001101&=(0,0,1,0)\\
		1000110&=(0,0,0,1) & 1001011&=(0,0,1,1) & 1011100&=(0,1,0,1) & 0010111&=(0,1,1,0)\\
		1010001&=(0,1,1,1) & 1110010&=(1,0,0,1) & 0111001&=(1,0,1,0) & 1111111&=(1,0,1,1)\\
		0101110&=(1,1,0,0) & 1101000&=(1,1,0,1) & 0100011&=(1,1,1,0) & 1100101&=(1,1,1,1)
	\end{array}
\]

Nótese que todas las coordenadas son distintas entre sí. Así, queda comprobado que $\mathcal{B}$ es un sistema generador de $S$ además de ser libre, por lo que podemos afirmar que es una base de $S$, que $S$ es un subespacio vectorial y que su dimensión es $\dim_{\mathbb{F}_2^7}S = 4$.