\chapter{Códigos Lineales}

\section{Ejercicio 1}

\begin{formulationBox}
	Sea $C\subseteq \mathbb{F}_q^n$ un código lineal de dimensión $k$ y distancia mínima $d$.
	
	\begin{enumerate}[label=\alph*)]
		\item Demostrar que $\sum_{i=0}^{\lfloor\frac{d-1}{2}\rfloor} \binom{n}{i} (q-1)^i \leq q^{n-k}$.
		\item ¿Existe un código lineal $C \subseteq \mathbb{F}_2^6$ con distancia mínima 3 y al menos 9 elementos? Razona la respuesta.
	\end{enumerate}
\end{formulationBox}

\begin{enumerate}[label=\alph*)]
	\item La cota de Hamming nos dice que
	\[|C|\sum_{i=0}^{\lfloor\frac{d-1}{2}\rfloor} \binom{n}{i} (q-1)^i \leq q^n.\]
	
	Por otra parte, ya que $C$ es un código lineal, es un subespacio, y por tanto se verifica que
	\[|C| = q^k.\]
	
	Sustituyendo en la inecuación, y dividiendo en ambos lados de la inecuación entre $q^k$, obtenemos
	\[\sum_{i=0}^{\lfloor\frac{d-1}{2}\rfloor} \binom{n}{i} (q-1)^i \leq q^{n-k}.\]
	
	\item Sustituyendo los valores en la inecuación:
	\[\sum_{i=0}^{\lfloor\frac{3-1}{2}\rfloor} \binom{6}{i} (2-1)^i \leq 2^{6-k} \implies 1 + 6 \leq 2^{6-k} \implies 7 \leq \frac{64}{2^k}.\]
	Hemos obtenido que $k \leq 3$. El número de elementos de $C$ es $|C| = q^k$, que es como mucho $2^3 = 8$, luego no existe un código lineal de al menos 9 elementos con dichas características.
\end{enumerate}

\section{Ejercicio 2}

\begin{formulationBox}
	Demostrar que los siguientes conjuntos son $(n, k)$-códigos lineales y determinar su dimensión, su distancia mínima y una matriz generadora:
	
	\begin{enumerate}[label=\alph*)]
		\item $C_1 = \{aa\hdots a\mid a\in\mathbb{F}_q\} \subseteq \mathbb{F}_q^n$ (Código de repetición)
		\item $C_2 = \{x_1\hdots x_n \mid x_i\in\mathbb{F}_q, i=1,\hdots,n,x_n = \sum_{i=1}^{n-1}x_i\}$ (Código de paridad)
	\end{enumerate}
\end{formulationBox}

Debemos comprobar si los conjuntos son subespacios vectoriales de $\mathbb{F}_q^n$.

\begin{enumerate}[label=\alph*)]
	\item Comprobamos si se cumplen las propiedades de los subespacios vectoriales:
		\begin{enumerate}[label=\arabic*)]
			\item El código $C_1$ no es vacío porque contiene exactamente una palabra por cada elemento del conjunto no vacío $\mathbb{F}_q$.
			\item Sean $\textbf{x} = aa\hdots a$ e $\textbf{y} = bb\hdots b$ tales que $a,b \in \mathbb{F}_q$, luego $\textbf{x}, \textbf{y} \in C_1$. Entonces, ya que $a+b, a-b \in \mathbb{F}_q$, se verifica $\textbf{x}+\textbf{y} = (a+b)(a+b)\hdots(a+b) \in C_1$.
			\item Para cualesquier $\lambda\in\mathbb{F}_q$, $\textbf{x}\in C_1$ con $\textbf{x} = a\hdots a$, se verifica $\lambda\textbf{x} = (\lambda a)\hdots(\lambda a) \in C_1$, ya que $\lambda a \in \mathbb{F}_q$.
		\end{enumerate}
		Por tanto, $C_1$ es un subespacio vectorial de $\mathbb{F}_q^n$.
		
		Es inmediato que, para cualquier $\textbf{x}\in C_1$, existe $a\in\mathbb{F}_q$ tal que $\textbf{x} = a\cdot 11\hdots1$, luego $\{1\hdots1\}$ es una base de $C_1$ y su dimensión es $k = 1$. Además, una matriz generadora de $C_1$ es $\begin{pmatrix}
			1 & 1 & \cdots & 1
		\end{pmatrix}$. Por último, su distancia mínima es $d(C_1) = n$, ya que todas las palabras de este código tienen peso $n$.
		
		Así, $C_1$ es un $(n, 1)$-código lineal.
	
	\item Comprobamos si se cumplen las propiedades de los subespacios vectoriales:
		\begin{enumerate}[label=\arabic*)]
			\item El código $C_2$ trivialmente no es vacío.
			\item Sean $\textbf{x} = x_1x_2\hdots x_n$ e $\textbf{y} = y_1y_2\hdots y_n$ con $\textbf{x}, \textbf{y} \in C_1$. Entonces, se verifica $\textbf{x}+\textbf{y} = (x_1+y_1)(x_2+y_2)\hdots(x_n+y_n)$. Teniendo en cuenta que $x_n = \sum_{i=1}^{n-1}x_i$ e $y_n = \sum_{i=1}^{n-1}y_i$, obtenemos que $x_n + y_n = \sum_{i=1}^{n-1}(x_i + y_n)$, luego $\textbf{x} + \textbf{y} \in C_2$.
			\item Para cualesquier $\lambda\in\mathbb{F}_q$, $\textbf{x}\in C_2$ con $\textbf{x} = x_1x_2\hdots x_n$, se verifica $\lambda\textbf{x} = (\lambda x_1)(\lambda x_2)\hdots(\lambda x_n) \in C_1$, ya que $\lambda x_n = \lambda\sum_{i=1}^{n-1}x_i = \sum_{i=1}^{n-1}\lambda x_i = (\lambda\textbf{x})_n$.
		\end{enumerate}
		Por tanto, $C_2$ es un subespacio vectorial de $\mathbb{F}_q^n$.
		
		Es inmediato que, para cualquier $\textbf{x}\in C_2$, existe $(\alpha_1, \alpha_2, \hdots, \alpha_{n-1})\in\mathbb{F}_q$ tal que
		\[\textbf{x} = \alpha_1\cdot10\hdots01 + \alpha_2\cdot01\hdots01 + \cdots + \alpha_{n-1}\cdot00\hdots11,\]
		luego $\{10\hdots01, 01\hdots01, \hdots, 00\hdots11\}$ es un sistema generador de $C_2$, y una base también puesto que es libre. Por tanto, la dimensión de $C_2$ es $k = n-1$, y una matriz generadora de $C_2$ es $\begin{pmatrix}
			1 & 0 & \cdots & 0 & 1 \\
			0 & 1 & \cdots & 0 & 1 \\
			\vdots & \vdots & \ddots & \vdots & \vdots \\
			0 & 0 & \cdots & 1 & 1
		\end{pmatrix}$. Por último, su distancia mínima es $d(C_2) = 2$, puesto que el peso mínimo es 2, como se aprecia con facilidad viendo la matriz generadora.
		
		Así, $C_2$ es un $(n, n-1)$-código lineal.
\end{enumerate}

\section{Ejercicio 3}

\begin{formulationBox}
	Sea $C \subseteq \mathbb{F}_3^3$ el código ternario con matriz generadora $G = \begin{pmatrix}
	2 & 1 & 1 \\
	1 & 2 & 0
	\end{pmatrix}$.
	
	\begin{enumerate}[label=\alph*)]
		\item Demostrar que no tiene ninguna matriz genneradora en forma estándar.
		\item Localizar un código lineal $C_1$ equivalente a $C$ que sí admita matriz generadora en forma estándar.
	\end{enumerate}
\end{formulationBox}

\begin{enumerate}[label=\alph*)]
	\item Para que $C$ pueda tener alguna matriz generadora en forma estándar, deben existir dos palabras $\textbf{x}, \textbf{y} \in C$ tales que $x_1 = 1$, $x_2 = 0$, $y_1 = 0$ e $y_2 = 1$.
	
	La matriz $G$ nos genera palabras de la forma $(2a + b)(a + 2b)a$, con $a,b \in \mathbb{F}_3$. Busquemos la palabra $\textbf{x}$ como la acabamos de definir. Entonces, necesitamos $2a + b = 1$ y $a + 2b = 0$.
	
	Si despejamos $b = 1 - 2a$, sustituimos en la segunda ecuación para obtener
	\[a + 2(1-2a) = a + 2 - 4a = -3a + 2 = 0,\]
	luego necesitamos que $3a = 2$; y, teniendo en cuenta que $a\in\mathbb{F}_3$, esto es equivalente a la ecuación $3a\mod3 = 2$. Esta ecuación no tiene solución, ya que $3a$ siempre será congruente con $0 \mod 3$ en $\mathbb{F}_3$.
	
	\item Si sumamos la fila 2 a la fila 1, obtenemos una matriz $G' = \begin{pmatrix}
		0 & 0 & 1 \\
		1 & 2 & 0
	\end{pmatrix}$. Ahora, mediante permutaciones de columnas, movemos la columna 3 a la 1, la 1 a la 2 y la 2 a la 3, y obtenemos una matriz en forma estándar: $G'' = \begin{pmatrix}
	1 & 0 & 0 \\
	0 & 1 & 2
	\end{pmatrix}$.
	
	Como las operaciones realizadas (\textit{sumar a una fila una combinación lineal de las restantes filas} y \textit{permutación de columnas}) producen matrices que generan códigos equivalentes, el código $C_1$ generado por la matriz $G''$ es equivalente al código $C$.
\end{enumerate}

\section{Ejercicio 4}

\begin{formulationBox}
	Sea $C \subseteq \mathbb{F}_q^n$ un código lineal y $\textbf{x}, \textbf{y}\in\mathbb{F}_q^n$. Se define la relación de equivalencia dada por:
	\[\textbf{x}\sim\textbf{y} \iff \textbf{x}-\textbf{y}\in C.\]
	Probar $\textbf{x}\sim\textbf{y}$ si y solo si $S(\textbf{x}) = S(\textbf{y})$, donde $S(\textbf{z})$ denota el síndrome de $\textbf{z}\in\mathbb{F}_q^n$.
\end{formulationBox}

En otras palabras, debemos probar que el método de decodificación basado en los líderes es equivalente al método de decodificación mediante síndromes.

Notemos que $S(\textbf{x}) - S(\textbf{y}) = \textbf{x}H^t - \textbf{y}H^t = (\textbf{x} - \textbf{y})H^t = S(\textbf{x}-\textbf{y})$. Por definición, una palabra $\textbf{z}$ está en $C$ si y solo si $\textbf{z}H^t = 0$. Entonces, $\textbf{x} - \textbf{y} \in C$ si y solo si $S(\textbf{x}-\textbf{y}) = S(\textbf{x}) - S(\textbf{y}) = 0$. Inmediatamente, \[\textbf{x}-\textbf{y} \in C \iff S(\textbf{x}) = S(\textbf{y}).\]

$\qed$

\section{Ejercicio 5}

\begin{formulationBox}
	Se considera el código lineal de $\mathbb{F}_3^4$ cuya matriz generadora es: $G = \begin{pmatrix}
		1 & 1 & 0 & 1 \\
		1 & 0 & 1 & 2
	\end{pmatrix}$.
	
	\begin{enumerate}[label=\alph*)]
		\item Calcular una matriz de control y su distancia mínima.
		\item Decodificar la palabra $2121$ empleando el método de los síndromes.
		\item ¿Tienen todas las palabras de $\mathbb{F}_3^4$ decodificación única? ¿Es un código perfecto? Razona la respuesta.
	\end{enumerate}
\end{formulationBox}

\begin{doubtBox}
	Si creamos una matriz $G_1$ en forma estándar, esta nueva matriz generará un código equivalente pero no igual al generado por $G$, ¿no?
\end{doubtBox}

\section{Ejercicio 6}

\begin{formulationBox}
	\begin{enumerate}[label=\alph*)]
		\item Construir un código de Hamming binario $C$ de longitud 7 y dimensión 4.
		\item Hallar la decodificación de la palabra: $1001010$, utilizando el método de decodificación basado en los líderes.
	\end{enumerate}
\end{formulationBox}

\begin{enumerate}[label=\alph*)]
	\item Recordemos que, en un código de Hamming, la longitud está dada por $\frac{q^r-1}{q-1}$ y la dimensión se da por $\frac{q^r-1}{q-1}-r$. Podemos calcular los parámetros deseados resolviendo las ecuaciones $\frac{q^r-1}{q-1} = 7$ y $\frac{q^r-1}{q-1}-r = 4$.
	
	Llamemos $l$ a la longitud. Entonces, tenemos $l = 7$ y $l-r = 4$, luego $r = 3$.
	
	Ahora, $l = \frac{q^3-1}{q-1} = 7 \implies q^3 - 7q = -6$, con $q \neq 1$. Las única solución natural es $q = 2$.
	
	Entonces, tenemos el cuerpo $\mathbb{F}_2^3$, que tiene 7 subespacios de dimensión 1. Construimos una matriz con los vectores de las bases de cada uno de estos subespacios:
	
	\[H = \begin{pmatrix}
		1 & 1 & 0 & 1 & 1 & 0 & 0 \\
		1 & 1 & 1 & 0 & 0 & 1 & 0 \\
		1 & 0 & 1 & 1 & 0 & 0 & 1
	\end{pmatrix}.\]
	
	Como $H$ es de la forma $(-B^t|I_{n-k})$, tenemos una matriz generadora para el código que buscamos:
	\[G = (I_k|B) = \begin{pmatrix}
		1 & 0 & 0 & 0 & 1 & 1 & 1 \\
		0 & 1 & 0 & 0 & 1 & 1 & 0 \\
		0 & 0 & 1 & 0 & 0 & 1 & 1 \\
		0 & 0 & 0 & 1 & 1 & 0 & 1
	\end{pmatrix}.\]
	
	Con asistencia del \textit{script} de Python \texttt{leader.py}, recogido en el apéndice B, vemos que la palabra $1001010$ se decodifica en $1001010$; es decir, no ocurrió ningún error en la transmisión.
\end{enumerate}

\section{Ejercicio 7}

\begin{formulationBox}
	Sea $G = \begin{pmatrix}
		0 & 1 & 1 & 1 & 1 \\
		1 & 0 & 0 & 1 & 0
	\end{pmatrix} \in \text{Mat}_{2\times5}(\mathbb{F}_2)$ la matriz generadora de un código lineal $C$.
	
	\begin{enumerate}[label=\alph*)]
		\item Demostrar que $C$ es de dimensión 2 y calcular todas las palabras de $C$.
		\item Hallar una matriz de control de paridad.
		\item Buscar líderes de las clases de equivalencia del conjunto cociente $\mathbb{F}_2^5/C$.
	\end{enumerate}
\end{formulationBox}

\begin{enumerate}[label=\alph*)]
	\item El código $G$ es de dimensión $2$ si y solo si la matriz $G$ es una matriz de dos filas independientes, lo cual es evidente que se cumple.
	\item Primero hallaremos una matriz $G'$ generadora de un código $C'$ equivalente a $C$, tal que $G'$ esté en forma estándar. Es decir, $G'$ será de la forma $(I_k|B)$.
	Podemos permutar las filas para obtener
	$G' = \begin{pmatrix}
		1 & 0 & 0 & 1 & 0 \\
		0 & 1 & 1 & 1 & 1
	\end{pmatrix}$, que es justamente la matriz que buscábamos.
	
	La matriz de control de paridad para $C$ será $H = (-B^t|I_{n-k}) = \begin{pmatrix}
		0 & 1 & 1 & 0 & 0 \\
		1 & 1 & 0 & 1 & 0 \\
		0 & 1 & 0 & 0 & 1
	\end{pmatrix}$.
	\item \hspace*{1mm}
	
	\begin{draftBox}
		Este ejercicio no está resuelto.
	\end{draftBox}
\end{enumerate}

\section{Ejercicio 9}

\begin{formulationBox}
	Sea $C$ un código de bloque sobre $\mathbb{F}_q$ de longitud $n$. Se llama \textbf{polinomio enumerador de pesos} de $C$ al polinomio
	\[W_C(x, y) = \sum_{i=0}^{n} a_i x^i y^{n-i}\textrm{, siendo }a_i = |\{c\in C | w(c) = i\}|.\]
	
	\begin{enumerate}[label=\alph*)]
		\item Demostrar que si $C$ es un código lineal, entonces el número de palabras de $C$ que se encuentran a distancia $i$ de $c \in C$ es $a_i$.
		\item Si $C \in \mathbb{F}_2^5$ es el código lineal cuya matriz generadora viene dada por $G = \begin{pmatrix}
			1 & 1 & 1 & 0 & 0 \\
			0 & 0 & 1 & 1 & 0 \\
			1 & 1 & 1 & 1 & 1
		\end{pmatrix}$, calcular su polinomio enumerador de pesos y el de su código dual.
	\end{enumerate}
\end{formulationBox}

\begin{enumerate}[label=\alph*)]
	\item Recordemos que sean $\textbf{x}, \textbf{y} \in \mathbb{F}_q^n$, se cumple $d(\textbf{x}, \textbf{y}) = w(\textbf{x}-\textbf{y})$.
	
	%Por consiguiente, la expresión $a_i = |\{c\in C | w(c) = i\}|$ hace referencia al número de palabras de $C$ que se encuentran a distancia $i$ de $\textbf{0}\in C$.
	
	%Tomamos ahora $c_1, c_2 \in C$ tal que $w(c_1) = i$ y $w(c_2) = k$. Entonces, 
	
	%Existen $a_i$ palabras tales que $w(c) = i$. Para cada una de estas, podemos hacer $c - c_2$, que está en $C$ y que está a distancia $w(c - c_2)$ de $c_2$.
	
	Por definición, existen exactamente $a_i$ palabras de $C$ cuyo peso es $i$. Llamamos $A_i$ al conjunto $\{c\in C | w(c) = i\}$.
	
	Tomamos una palabra $c \in C$. Para que una palabra $c' \in C$ esté a distancia $i$ de $c$, debe cumplirse $w(c - c') = i$, luego $c - c' \in A_i$. Así, los valores posibles para $c'$ son las palabras $c - c_k$ tales que $c_k \in A_i$, ya que $c - (c - c_k) = c_k$.
	
	Por tanto, existen exactamente $a_i$ palabras que se encuentran a distancia $i$ de $c \in C$.
	
	$\qed$
	
	%y las palabras $c - c_k$ con $k \in [1, a_i]\cap\naturals$. Todas estas palabras $c - c_k$ están a distancia $i$ de $c$, ya que $d(c, c - c_k) = w(c - (c - c_k)) = w(c_k) = i$.
	
	\item Expresamos $C$ como el conjunto de sus palabras:
	\[C = \{00000, 11100, 00110, 11010, 11111, 00011, 11001, 00101\}.\]
	Así, obtenemos $a_0 = 1, a_1 = 0, a_2 = 3, a_3 = 3, a_4 = 0, a_5 = 1$, y su polinomio enumerador de pesos será
	\[W_C(x, y) = y^5 + 3x^2y^3 + 3x^3y^2 + x^5.\]
	
	Calculemos ahora el código dual $C^\perp$. Sabemos que $\textbf{x}\in C^\perp$ si y solo si $\textbf{x}G^t = 0$. Así, tenemos la siguiente ecuación:
	\begin{align*}
		\begin{pmatrix}
			x_1 & x_2 & x_3 & x_4 & x_5
		\end{pmatrix}\begin{pmatrix}
			1 & 0 & 1 \\
			1 & 0 & 1 \\
			1 & 1 & 1 \\
			0 & 1 & 1 \\
			0 & 0 & 1
		\end{pmatrix} &= \\
		\begin{pmatrix}
			x_1 + x_2 + x_3 & x_3 + x_4 & x_1 + x_2 + x_3 + x_4 + x_5
		\end{pmatrix} &= \begin{pmatrix}
			0 & 0 & 0
		\end{pmatrix},
	\end{align*}
	de donde obtenemos las restricciones $x_3 = -x_4$, $x_3 = -x_1 - x_2$ y $x_5 = x_3$. Dicho de otra forma, las palabras de $C^\perp$ son de la forma $\begin{pmatrix}
		\alpha & \beta & -\alpha-\beta & \alpha+\beta & -\alpha-\beta
	\end{pmatrix}$.

	Por tanto, podemos expresar $C^\perp$ como el conjunto de sus palabras de la siguiente forma:
	
	\[C^\perp = \{00000, 10111, 01111, 11000\}.\]
	
	Así, obtenemos $a_0 = 1, a_1 = 0, a_2 = 1, a_3 = 0, a_4 = 2, a_5 = 0$, y su polinomio enumerador de pesos será
	\[W_C(x, y) = y^5 + x^2y^3 + 2x^4y.\]
	
\end{enumerate}

\section{Ejercicio 10}

\begin{formulationBox}
	Sea $C\subseteq \mathbb{F}_q^n$ un $(n,k)$-código lineal y $W_C(x,y)$ su polinomio enumerador de pesos. Demostrar que:
	\begin{enumerate}[label=\alph*)]
		\item $W_C(1,1) = q^k$.
		\item $W_C(0,1) = 1$.
		\item Si $q = 2$, entonces $W_C(1,0)\in\{0, 1\}$.
		\item Si $q = 2$, entonces $W_C(x,y) = W_C(y, x)$ si y solo si $W_C(1,0) = 1$.
	\end{enumerate}
\end{formulationBox}

\begin{enumerate}[label=\alph*)]
	\item Si damos valor $1$ tanto a la variable $x$ como a la $y$, obtendremos la suma de todos los $a_i$ con $i \in [0, n]\cap\naturals$.\footnote{Entendiendo que $0 \in \naturals$.} Está claro que todas las palabras tienen un peso en dicho intervalo, luego la suma dará como resultado el número de palabras en el código $C$ que, al tratarse de un código lineal, será igual a $q^k$.
	\item Si se anula la variable $x$, se anularán todos los sumandos menos $a_0y^n$. Si, además, damos valor $1$ a la variable $y$, obtenemos $W_C(0,1) = a_0$. Trivialmente, siempre existe una única palabra con peso igual a $0$, luego $a_0 = 0$.
	\item Si se anula la variable $y$ y se le da valor $1$ a la variable $x$, obtenemos $W_C(1,0) = a_n$. Con $q=2$, existe una única palabra posible con peso igual a $n$, la palabra $11\dots11$, por lo que $a_n$ solo puede tener $0$ o $1$ como valor.
	\item Recordemos que, en un código lineal, sean $\textbf{x}, \textbf{y} \in C$, se cumple $\textbf{x}-\textbf{y} \in C$.
	
	Tomamos $c \in C$. Si $11\dots11 \in C$, entonces podemos calcular $c' = 11\dots11 - c$, que cumple $w(c') = n - w(c)$. Por tanto, $a_i = a_{n-i}$ para todo $i$ en su dominio, y $W_C(1,0) = 1 \implies W_C(x,y) = W_C(y,x)$.
	
	Por otra parte, en caso de que $W_C(x,y) = W_C(y,x)$, evidentemente se cumplirá $W_C(0,1) = W_C(1, 0)$. Ya hemos demostrado que $W_C(0,1) = 1$, por lo que $W_C(x,y) = W_C(y,x) \implies W_C(1,0) = 1$.
\end{enumerate}

\section{Ejercicio 11}

\begin{formulationBox}
	Sea $C \subseteq \mathbb{F}_2^n$ un código lineal. Demostrar que se verifica una de las dos afirmaciones siguientes:
	\begin{enumerate}[label=\alph*)]
		\item Todas las palabras son de peso par.
		\item La mitad de las palabras son de peso par y la otra mitad de peso impar.
	\end{enumerate}
\end{formulationBox}

%Existen códigos lineales cuyas palabras son todas de peso par: por ejemplo, el código trivial $C = \{\textbf{0}\}$. Por tanto, debemos demostrar que, en caso de la primera afirmación no se cumpla, entonces la segunda debe ser verdadera.

Llamamos $P(n)$ a la proposición que dice que en un código de dimensión $n$ se verifica alguna de las dos afirmaciones.

Notemos que, con $q = 2$ (como es el caso), el peso de una palabra equivale a la suma de sus letras.

Siendo $C$ un código lineal, debe poder construirse mediante una base de palabras linealmente independientes. Esto nos permite demostrar $P(n)\forall n\in\naturals$ mediante inducción: demostraremos que, dado un código lineal cualquiera, añadiendo una nueva palabra linealmente independiente a la base construiremos un nuevo código lineal que cumpla alguna de las dos afirmaciones.

\begin{itemize}
	\item \textbf{Base}: cuando el número de palabras en la base es igual a $0$, la base $\{\}$ genera el código trivial $C_0 = \{\textbf{0}\}$, que cumple la primera afirmación. Por tanto, $P(0)$ es verdadera.
	\item \textbf{Paso inductivo}: demostraremos que, si $P(k)$ es verdadera para algún $k\in\naturals$, entonces $P(k+1)$ es verdadera también.
	
	Sea $C_k$ un código de dimensión $k$. Si añadimos a su base una nueva palabra linealmente independiente, $\textbf{x}_{k+1}$, por la definición de código lineal, el nuevo código $C_{k+1}$ contendrá todas las palabras $\textbf{x}_{k+1} + \textbf{y}$, con $\textbf{y} \in C_k$. existen los dos siguientes supuestos:
	
	\begin{enumerate}
		\item Si $\textbf{x}_{k+1}$ es par, entonces la paridad de $w(\textbf{x}_{k+1} + \textbf{y})$ será la misma que la de $w(\textbf{y})$ para todos los elementos $\textbf{y} \in C_k$, luego $C_{k+1}$ tendrá el doble de palabras pares (así como impares) que $C_k$.
		
		Como consecuencia, la proporción de palabras pares en $C_{k+1}$ será la misma que la de $C_k$: y, como $P(k)$ es verdadera, significa que en el caso de $C_{k+1}$ también serán pares todas o exactamente la mitad de las palabras, luego se verificará $P(k+1)$.
		
		\item Si $\textbf{x}_{k+1}$ es impar, entonces la paridad de $w(\textbf{x}_{k+1} + \textbf{y})$ será la opuesta que la de $w(\textbf{y})$ para todos los elementos $\textbf{y} \in C_k$.
		
		Así, si las $2^k$ palabras en $C_k$ tienen peso par, se introducen $2^k$ palabras con peso impar en $C_{k+1}$, obteniendo un código de $2^{k+1}$ palabras de las cuales la mitad son de peso par y la otra mitad de peso impar.
		
		Similarmente, si $C_k$ tiene $2^{k-1}$ palabras de peso par y $2^{k-1}$ palabras de peso impar, se introducen la misma cantidad de palabras de peso par e impar en $C_{k+1}$, dando lugar a un código de $2^k$ palabras pares y $2^k$ palabras impares, verificando así $P(k+1)$.
	\end{enumerate}
\end{itemize}

$\qed$

\section{Ejercicio 12}

\begin{formulationBox}
	Sea $C \subseteq \mathbb{F}_2^n$ un código lineal. Demostrar que se verifica una de las dos afirmaciones siguientes:
	\begin{enumerate}[label=\alph*)]
		\item Todas las palabras empiezan por 0.
		\item La mitad de las palabras empiezan por 0 y la otra mitad por 1.
	\end{enumerate}
\end{formulationBox}

La demostración es casi idéntica que la del ejercicio 11. En este caso, en lugar de las propiedades de la paridad, debemos tener en cuenta las cuatro siguientes:

\begin{itemize}
	\item Si $\textbf{x}$ empieza por 0 y $\textbf{y}$ empieza por 0, entonces $\textbf{x}+\textbf{y}$ empieza por 0.
	\item Si $\textbf{x}$ empieza por 0 y $\textbf{y}$ empieza por 1, entonces $\textbf{x}+\textbf{y}$ empieza por 1.
	\item Si $\textbf{x}$ empieza por 1 y $\textbf{y}$ empieza por 0, entonces $\textbf{x}+\textbf{y}$ empieza por 1.
	\item Si $\textbf{x}$ empieza por 1 y $\textbf{y}$ empieza por 1, entonces $\textbf{x}+\textbf{y}$ empieza por 0.
\end{itemize}

Notemos que podríamos haber enumerado unas propiedades de manera similar para el ejercicio 11:

\begin{itemize}
	\item Si $\textbf{x}$ es de peso par y $\textbf{y}$ es de peso par, entonces $\textbf{x}+\textbf{y}$ es de peso par.
	\item Si $\textbf{x}$ es de peso par y $\textbf{y}$ es de peso impar, entonces $\textbf{x}+\textbf{y}$ es de peso impar.
	\item Si $\textbf{x}$ es de peso impar y $\textbf{y}$ es de peso par, entonces $\textbf{x}+\textbf{y}$ es de peso impar.
	\item Si $\textbf{x}$ es de peso impar y $\textbf{y}$ es de peso impar, entonces $\textbf{x}+\textbf{y}$ es de peso par.
\end{itemize}

Con esto, adaptar la demostración es una tarea trivial.

\section{Ejercicio 13}

\begin{formulationBox}
	Sea $C \subseteq \mathbb{F}_q^n$ un $(n, k)$-código lineal y $C^\perp$ su código dual. Se dice que $C$ es autoortogonal si $C \subseteq C^\perp$ y $C$ es autodual si $C = C^\perp$. Demostrar que $C$ es autodual si y solo si $C$ es autoortogonal y $\dim C = \frac{n}{2}$.
\end{formulationBox}

Recordemos que, dado un $(n, k)$-código lineal $C$, la dimensión de su código dual es $\dim C^\perp = n-k$.

Si $C$ es autodual, $C$ y $C^\perp$ tienen los mismos elementos, luego los elementos de $C$ pertenecen también a $C^\perp$, por lo que $C$ también es autoortogonal. Además, si $C$ y $C^\perp$ tienen exactamente los mismos elementos, su dimensión debe ser la misma, luego se verifica $k = n-k$, y de aquí se obtiene $k = \dim C = \frac{n}{2}$.

Similarmente, si $\dim C = \frac{n}{2}$, entonces $\dim C^\perp = n - \frac{n}{2} = \frac{n}{2} = k = \dim C$, luego $C$ y $C^\perp$ tienen la misma cantidad de elementos. Además, si $C$ es autoortogonal, entonces todos los elementos de $C$ pertenecen también a $C^\perp$; y, como ambos códigos tienen la misma cantidad de elementos, $C^\perp$ únicamente contrendrá los elementos de $C$, por lo que $C$ será autodual.

$\qed$

\section{Ejercicio 14}

\begin{formulationBox}
	Sea $C_i^\perp$ el código dual del código lineal $C_i, i=1, 2.$ Demostrar que:
	\begin{enumerate}[label=\alph*)]
		\item $(C_i^\perp)^\perp = C_i$.
		\item $(C_1 + C_2)^\perp = C_1^\perp \cap C_2^\perp$.
	\end{enumerate}
\end{formulationBox}

Recordemos la definición de código dual:
\[C^\perp = \{\textbf{x} \in \mathbb{F}_q^n | \textbf{x}\cdot\textbf{y} = 0, \forall\textbf{y} \in C\}.\]

\begin{enumerate}[label=\alph*)]
	\item Por definición, $C_i^\perp$ contiene las palabras cuyo producto escalar con todos los elementos de $C_i$ es igual a 0.
	
	Similarmente, $(C_i^\perp)^\perp$ contiene las palabras cuyo producto escalar con todos los elementos de $C_i^\perp$ es igual a 0. Por cómo está construido $C_i^\perp$, y ya que el producto escalar es conmutativo, estas palabras son precisamente las de $C_i$, luego $(C_i^\perp)^\perp$ tiene las mismas palabras que $C_i$ y, por tanto, $(C_i^\perp)^\perp = C_i$.
	
	\item Por definición, $(C_1 + C_2)^\perp$ contiene las palabras cuyo producto escalar con todos los elementos de $C_1 + C_2$ es igual a 0. El código $C_1 + C_2$ contiene palabras tanto de $C_1$ como de $C_2$. Por tanto, $(C_1 + C_2)^\perp$ contendrá palabras cuyo producto escalar con todos los elementos de $C_1$ y de $C_2$ sea igual a 0. Por tanto, estas palabras deberán estar incluidas tanto en $C_1^\perp$ como en $C_2^\perp$: dicho de otra manera, estarán incluidas en $C_1^\perp \cap C_2^\perp$.
	
	Similarmente, si una palabra está incluida tanto en $C_1^\perp$ como en $C_2^\perp$, significa que su producto escalar con cualquier palabra de $C_1 + C_2$ será 0, luego también estará incluida en $(C_1 + C_2)^\perp$.
\end{enumerate}

$\qed$

\section{Ejercicio 15}

\begin{formulationBox}
	Para $i = 1, 2$, consideraremos $C_i \in \mathbb{F}_2^{n_i}$ código lineal de dimensión $k$, distancia mínima $d_i$ y matriz generadora $G_i$.
	\begin{enumerate}[label=\alph*)]
		\item Demostrar que el código lineal con matriz generadora $\begin{pmatrix}
			G_1 & 0\\
			0 & G_2
		\end{pmatrix}$ tiene longitud $n_1 + n_2$, dimensión $2k$ y distancia mínima $d = \min\{d_1, d_2\}$.
		\item Demostrar que el código lineal con matriz generadora $\begin{pmatrix}
			G_1 & G_2
		\end{pmatrix}$ tiene longitud $n_1 + n_2$, dimensión $k$ y distancia mínima $d \geq d_1 + d_2$.
	\end{enumerate}
\end{formulationBox}

Llamemos $C_{a)}$ al código del que se habla en el apartado a) y $C_{b)}$ al código del que se habla en el apartado b).

\begin{enumerate}[label=\alph*)]
	\item La longitud de un código es igual al número de columnas de su matriz generadora. La matriz generadora de $C_{a)}$ tiene una columna por cada una de $G_1$, seguidas de una columna por cada una de $G_2$, luego en total habrá $n_1 + n_2$ columnas.
	
	La dimensión de un código lineal está dada por el número de palabras en la base. La base de $C_{a)}$ está formada de $k$ palabras provenientes de la base de $C_1$ (con terminación $00\hdots00$) y $k$ palabras provenientes de la base de $C_2$ (con prefijo $00\hdots00$), luego en total tiene $2k$ palabras.
	
	La distancia mínima de un código lineal es igual a su peso mínimo. El hecho de que las palabras de la base tengan una terminación o un prefijo de letras nulas significa que tenemos en $C_{a)}$ palabras con un peso tan bajo como las de $C_1$ o $C_2$. Así, la palabra de peso mínimo en $C_{a)}$ tendrá el mismo peso que la palabra de $C_1$ o $C_2$ que tenga que menor peso.
	
	\item La longitud sigue el mismo razonamiento que en el apartado a).
	
	En este caso, la base de $C_{b)}$ está formada de $k$ palabras en total, resultantes de concatenar las palabras de la base de $C_1$ con las de la base de $C_2$.
	
	Recordemos que la distancia mínima de un código lineal equivale a su peso mínimo. Ya que las palabras de $C_{b)}$ son concatenación de palabras de $C_1$ y $C_2$, el peso de una palabra de $C_{b)}$ será igual a la suma del peso de una palabra de $C_1$ y otra de $C_2$. Por tanto, el peso mínimo de $C_{b)}$ debe ser como mínimo igual a la suma del peso mínimo de $C_1$ y de $C_2$.
\end{enumerate}

$\qed$

\section{Ejercicio 16}

\begin{formulationBox}
	Demostrar que si un código lineal admite una matriz generadora en forma estándar, entonces esta es única.
\end{formulationBox}

Supongamos que existe un código lineal $C$ que puede ser generado por dos matrices en forma estándar diferentes: $G_1 = (I_k|B_1)$ y $G_2 = (I_k|B_2)$. Entonces, el código generado por $G_1$ debe incluir las palabras de las filas que conforman $G_2$, pues son en realidad el mismo código.

Llamemos $f$ a la fila (o una de las filas) en la que se distinguen $G_1$ y $G_2$. En dicha fila, $G_1$ tiene la palabra
\[\textbf{x}_1 = \underbrace{0\hdots0}_\text{$f-1$}1\underbrace{0\hdots0}_\text{$k-f$}a_1b_1c_1\hdots,\]
mientras que $G_2$ tiene la palabra
\[\textbf{x}_2 = \underbrace{0\hdots0}_\text{$f-1$}1\underbrace{0\hdots0}_\text{$k-f$}a_2b_2c_2\hdots.\]

Si intentamos escribir $\textbf{x}_2$ como combinación lineal de las palabras de $G_1$, veremos que solo es posible en caso de que $\textbf{x}_1 = \textbf{x}_2$, lo que significaría que $G_1$ y $G_2$ son en realidad la misma matriz, luego llegamos a una contradicción.

Por tanto, un código lineal no puede ser generado por dos matrices en forma estándar diferentes.

$\qed$
