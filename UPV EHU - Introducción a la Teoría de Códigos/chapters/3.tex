\chapter{Códigos Lineales}

\section{Ejercicio 1}

\begin{formulationBox}
	Sea $C\subseteq \mathbb{F}_q^n$ un código lineal de dimensión $k$ y distancia mínima $d$.
	
	\begin{enumerate}[label=\alph*)]
		\item Demostrar que $\sum_{i=0}^{\lfloor\frac{d-1}{2}\rfloor} \binom{n}{i} (q-1)^i \leq q^{n-k}$.
		\item ¿Existe un código lineal $C \subseteq \mathbb{F}_2^6$ con distancia mínima 3 y al menos 9 elementos? Razona la respuesta.
	\end{enumerate}
\end{formulationBox}

\begin{enumerate}[label=\alph*)]
	\item La cota de Hamming nos dice que
	\[|C|\sum_{i=0}^{\lfloor\frac{d-1}{2}\rfloor} \binom{n}{i} (q-1)^i \leq q^n.\]
	
	Por otra parte, ya que $C$ es un código lineal, es un subespacio, y por tanto se verifica que
	\[|C| = q^k.\]
	
	Sustituyendo en la inecuación, y dividiendo en ambos lados de la inecuación entre $q^k$, obtenemos
	\[\sum_{i=0}^{\lfloor\frac{d-1}{2}\rfloor} \binom{n}{i} (q-1)^i \leq q^{n-k}.\]
	
	\item Sustituyendo los valores en la inecuación:
	\[\sum_{i=0}^{\lfloor\frac{3-1}{2}\rfloor} \binom{6}{i} (2-1)^i \leq 2^{6-k} \implies 1 + 6 \leq 2^{6-k} \implies 7 \leq \frac{64}{2^k}.\]
	Hemos obtenido que $k \leq 3$. El número de elementos de $C$ es $|C| = q^k$, que es como mucho $2^3 = 8$, luego no existe un código lineal de al menos 9 elementos con dichas características.
\end{enumerate}
