\chapter{Nociones básicas de la Teoría de Códigos}

\section{Ejercicio 1}

\begin{formulationBox}
	Estudiar si las siguientes tuplas corresponden a un código EAN:
	\begin{enumerate}[label=\alph*)]
		\item 9783540283713
		\item 8412345678914
		\item 9783662479735
	\end{enumerate}
\end{formulationBox}

Debemos estudiar si se satisface la siguiente ecuación:
\[3\sum_{i=0}^5a_{2i+1} + \sum_{i=0}^6a_{2i} \equiv 0\mod10.\]

\begin{enumerate}[label=\alph*)]
	\item
	\begin{align*}
		3\sum_{i=0}^5a_{2i+1} + \sum_{i=0}^6a_{2i} & = 3(7+3+4+2+3+1) + (9+8+5+0+8+7+3)\\
		&= 60 + 40\\
		&= 100 \equiv 0 \mod 10,
	\end{align*}
	por lo que $9783540283713$ es un código EAN.
	
	\item
	\begin{align*}
		3\sum_{i=0}^5a_{2i+1} + \sum_{i=0}^6a_{2i} & = 3(4+2+4+6+8+1) + (8+1+3+5+7+9+4)\\
		&= 75 + 37\\
		&= 112 \not\equiv 0 \mod 10,
	\end{align*}
	por lo que $8412345678914$ no es un código EAN.
	
	\item
	\begin{align*}
		3\sum_{i=0}^5a_{2i+1} + \sum_{i=0}^6a_{2i} & = 3(7+3+6+4+9+3) + (9+8+6+2+7+7+5)\\
		&= 96 + 64\\
		&= 110 \equiv 0 \mod 10,
	\end{align*}
	por lo que $9783662479735$ es un código EAN.
\end{enumerate}

\section{Ejercicio 2}

\begin{formulationBox}
	Determinar el valor de $a$ para que las siguientes tuplas correspondan a un código EAN:
	\begin{enumerate}[label=\alph*)]
		\item 843a554161836
		\item 4325351455a52
		\item 978421345667a
	\end{enumerate}
\end{formulationBox}

\begin{enumerate}[label=\alph*)]
	\item
	\begin{align*}
		3\sum_{i=0}^5a_{2i+1} + \sum_{i=0}^6a_{2i} & = 3(4+a+5+1+1+3) + (8+3+5+4+6+8+6)\\
		&= 3a + 42 + 40\\
		&= 3a + 82 \equiv 0 \mod 10 \implies 3a \equiv 8 \mod 10 \implies a = 6.
	\end{align*}
	resultando en el código EAN $8436554161836$.
	
	\item
	\begin{align*}
		3\sum_{i=0}^5a_{2i+1} + \sum_{i=0}^6a_{2i} & = 3(3+5+5+4+5+5) + (4+2+3+1+5+a+2)\\
		&= 81 + 17 + a\\
		&= 98 + a \equiv 0 \mod 10 \implies a \equiv 2 \mod 10 \implies a = 2,
	\end{align*}
	resultando en el código EAN $4325351455252$.
	
	\item
	\begin{align*}
		3\sum_{i=0}^5a_{2i+1} + \sum_{i=0}^6a_{2i} & = 3(7+4+1+4+6+7) + (9+8+2+3+5+6+a)\\
		&= 87 + 33 + a\\
		&= 120 + a \equiv 0 \mod 10 \implies a \equiv 0 \mod 10 \implies a = 0,
	\end{align*}
	resultando en el código EAN $9784213456670$.
\end{enumerate}

\section{Ejercicio 3}

\begin{formulationBox}
	Determinar la distancia mínima del código EAN. Deducir cuántos errores detecta y corrige. Probar que el código detecta la transposición de dos dígitos consecutivos; es decir, del cambio de la palabra $a_0\dots a_ia_{i+1}a_{12}$ del código por $a_0\dots a_{i+1}a_i\dots a_{12}$, con $i\in\{0,\dots,11\}$.
\end{formulationBox}

Si tomamos una palabra del código EAN y le cambiamos un solo dígito, siempre nos veremos obligados a cambiar uno más para compensarlo y que vuelva a satisfacerse la ecuación. Por lo tanto, no hay palabras en el código que difieran entre sí en un solo dígito, mas sí en $2$, por lo que la distancia mínima del código EAN es $d(C_{EAN})=2$.

Siguiendo el mismo razonamiento, si tomamos una palabra $\textbf{c}\in C_{EAN}$ y le cambiamos un solo dígito, obteniendo una nueva palabra $\textbf{y}$, la ecuación no va a satisfacerse para $\textbf{y}$, por lo que podremos detectar el error. Esto no ocurre cuando hay más de un cambio. Por tanto, el código EAN detecta hasta 1 error.

Sin embargo, si recibimos la palabra $\textbf{y}$, hay múltiples palabras $\textbf{c}\in C_{EAN}$ tales que $d(\textbf{c}, \textbf{y}) = 1$; es decir, la palabra $\textbf{y}$ no tiene una decodificación única, luego el código EAN no corrige ningún error.

Al contrario de lo que insinúa el enunciado, el código EAN no siempre es capaz de detectar la transposición de dos dígitos consecutivos. Por ejemplo, si tomamos la palabra $\textbf{c} = 8771493526675$ y le aplicamos la transposición a los dígitos consecutivos $4$ y $9$, obtenemos la palabra $\textbf{c}' = 8771943526675$, que ¡también pertenece a $C_{EAN}$! y, por lo tanto, no va a ser detectado el error.

\section{Ejercicio 4}

\begin{formulationBox}
	Sea $C$ un código de bloque sobre $\mathbb{F}_q$ de longitud $n$. Se llama \textit{polinomio enumerador de pesos} de $C$ al polinomio
	\[W_C(x,y) = \sum_{i=0}^n a_i x^i y^{n-i}, \quad \text{siendo } a_i = |\{c \in C \mid w(c) = i\}|.\]
	
	Calcular $W_C(x,y)$ para los siguientes códigos:
	
	\begin{enumerate}[label=\alph*)]
		\item $C_1 = \{aa\dots a \mid a\in\mathbb{F}_q\} \subseteq \mathbb{F}_q^n$ (Código de repetición)
		\item $C_2 = \{x_1\dots x_n \mid x_i\in\mathbb{F}_q, i = 1, \dots, n, x_n = \sum_{i=1}^{n-1} x_i \}$ (Código de paridad)
	\end{enumerate}
\end{formulationBox}

\begin{enumerate}[label=\alph*)]
	\item El código $C_1$ tiene $q$ palabras distintas, cada una de las cuales tiene exactamente $n$ componentes idénticos.
	
	En el caso de la palabra $\mathbf{x_0} = 00\dots0$, no hay componentes no nulas, por lo que $w(\mathbf{x_0}) = 0$. Es el único caso en que ocurre esto, luego, según la definición dada para el polinomio enumerador de pesos, $a_0 = 1$.
	
	Por otra parte, todas las palabras $aa\dots a$ tales que $a\neq0$ tienen un peso de $n$. Como hay $q-1$ palabras que cumplen esto, $a_n = q-1$.
	
	Por último, $a_i = 0$ si y solo si $i \notin \{0, n\}$.
	
	Así, obtenemos el polinomio
	\[W_{C_1}(x, y) = \sum_{i=0}^n a_i x^i y^{n-i} = y^n + (q-1)x^n.\]
	
	\item \hspace*{1mm}
	
	\begin{draftBox}
		Este ejercicio no está del todo resuelto. El texto a continuación es tan solo un borrador.
	\end{draftBox}
	
	En el código $C_2$, las palabras pueden tener cualquier secuencia de $n-1$ elementos de $\mathbb{F}_q$, seguidas siempre de un último elemento (la llamaremos \textit{letra de paridad}) definido como la suma de los anteriores.
	
	%La única manera de que el peso de una palabra sea $0$ es que todas las letras sean $0$, luego $a_0 = 1$.
	
	%No hay ninguna palabra cuyo peso sea $1$; si se decidiera cambiar un único componente, también se acabaría cambiando la letra de paridad también. Por tanto, $a_1 = 0$.
	
	%Vamos a estudiar el código de paridad binario por simplificar los cálculos.
	
	Es fácil ver que la dimensión del código es $n-1$, pues la letra de paridad es linealmente dependiente de las demás.
	
	Intentando resolver el problema, escribí un código en el lenguaje de programación Python, en un fichero llamado \texttt{counter.py} (véase el apéndice A). Este \textit{script} calcula por fuerza bruta todos los $a_i$ para cualesquiera (pero no muy grandes) valores de $q$ y $n$. Con ayuda de la OEIS \cite{OEIS:A109499},
	
	Para facilitar la visualización de las palabras de $C_2$, vamos a hacer uso de la Teoría de Grafos. Vamos a comenzar estudiando las palabras con peso igual a $n$ (es decir, que no contengan ningún $0$).
	
	Visualizaremos el conjunto de dichas palabras de $C_2$ como un grafo completo de $q$ vértices (numerados de $0$ a $q-1$), y cada una de esas palabras palabras como caminos de longitud $n-1$ en dicho grafo, de la siguiente manera: el valor de una letra $x_i$ representa una arista desde un vértice hasta el vértice que se encuentra $x_i$ a la derecha; y el valor $x_{n-1}$ representa el vértice final del camino.
	
	Por ejemplo, la palabra $32132$ del código $C_2$ con $q=7$ y $n=5$ se puede entender como el siguiente camino:
	
	\begin{center}
		\begin{tikzpicture}
			\graph[n=7, counterclockwise, radius=3cm, edge={secondaryEdgeColor}]
			{
				subgraph K_n [V={0,1,2,3,4,5,6}];
			};
			\foreach \source/\dest in {0/3, 3/5, 5/6, 6/2}
			{
				\draw[primaryEdgeColor, ultra thick] (\source) -- (\dest);
			}
		\end{tikzpicture}
	\end{center}
	\[0 \xrightarrow{+3} 3 \xrightarrow{+2} 5 \xrightarrow{+1} 6 \xrightarrow{+3} 2.\]
	
	Notemos que podemos conseguir un camino cerrado si añadimos una arista desde el último vértice a $0$. Por ejemplo, en el anterior ejemplo, obtendríamos el siguiente camino cerrado:
	
	\begin{center}
		\begin{tikzpicture}
			\graph[n=7, counterclockwise, radius=3cm, edge={secondaryEdgeColor}]
			{
				subgraph K_n [V={0,1,2,3,4,5,6}];
			};
			\foreach \source/\dest in {0/3, 3/5, 5/6, 6/2, 2/0}
			{
				\draw[primaryEdgeColor, ultra thick] (\source) -- (\dest);
			}
		\end{tikzpicture}
	\end{center}
	\[0 \xrightarrow{+3} 3 \xrightarrow{+2} 5 \xrightarrow{+1} 6 \xrightarrow{+3} 2 \xrightarrow{-2} 0.\]
	
	Según nuestra interpretación, un $0$ significaría un movimiento al mismo vértice; pero esto está prohibido por nuestra definición del grafo, puesto que no existen aristas con el mismo vértice de salida que de llegada.
	
	Así, queda claro que averiguar el número de palabras con peso igual a $n$ es equivalente a hallar el número de caminos cerrados de longitud $n$ en un grafo completo de $q$ vértices.
	
	Para esto, primero sacaremos la matriz de adyacencia $A$ del grafo completo, que no es más que una matriz de $q\times q$ con todos los elementos iguales a $1$ salvo los de la diagonal, que están a $0$. Visto de otra manera, $A = J-I$, siendo $J$ la matriz con todos los elementos a $1$ e $I$ la matriz de identidad.
	
	La matriz $J$ tiene un único autovalor no nulo, que tiene valor $q$, mientras que los demás autovalores son $0$. Al restarle la matriz identidad, los autovalores disminuyen en $1$, por lo que $A$ tiene un autovalor $q-1$ con multiplicidad $1$ y autovalores $-1$ con multiplicidad $q-1$ \cite{fox2009}.
	
	Esto nos es útil ya que el número de caminos cerrados de longitud $n$ es igual a la traza de $A^n$, que a su vez está dada por la suma de los autovalores de $A^n$ (que son iguales a los autovalores de $A$ elevados a $n$); es decir, $\trace(A^n) = (q-1)^n + (-1)^n(q-1)$.
	
	Este valor es el número de caminos cerrados de longitud $n$ partiendo de cualquier nodo. Como hay $q$ nodos, para obtener la cantidad de dichos caminos partiendo de un nodo en concreto, podemos dividir la traza de antes entre $q$. De esta forma, obtendremos la expresión que buscábamos originalmente:
	
	\[a_n = \frac{(q-1)^n + (-1)^n(q-1)}{q}.\]
\end{enumerate}
	
\section{Ejercicio 5}

\begin{formulationBox}
	Sea $A = \{a_1,\hdots,a_m\}$ un alfabeto, $T_n = \{x_1\hdots x_n|x_i\in A, i=1,\hdots,n\}$ y $d:T_n\times T_n\rightarrow\{0,1,2,\hdots,n\}$ la aplicación definida por
	\[d(x_1\hdots x_n, y_1\hdots y_n) = |\{i\in\{1,\hdots,n\}|x_i\neq y_i\}|,\]
	para todo $x_1\hdots x_n, y_1\hdots y_n\in T_n$. Demostrar que $d$ es una distancia.
\end{formulationBox}

Dicho de otra manera, la distancia $d(\textbf{x}, \textbf{y})$ es el número de letras que las palabras en que difieren $\textbf{x}$ e $\textbf{y}$ (distancia de Hamming).

Primero, recordemos las propiedades de las distancias:

\begin{enumerate}[label=\alph*)]
	\item La distancia de un punto a sí mismo es $0$.
	\item La distancia entre dos puntos distintos es positiva.
	\item La distancia de $x$ a $y$ es la misma que de $y$ a $x$.
	\item Se cumple la desigualdad triangular:
	\[d(x, y) \leq d(x, z) + d(z, y).\]
\end{enumerate}

Ahora, verificamos que se cumplen para la aplicación definida en el enunciado:

\begin{enumerate}[label=\alph*)]
	\item En efecto, $d(x_1\hdots x_n, x_1\hdots x_n) = |\{i\in\{1,\hdots,n\}|x_i\neq x_i\}| = |\{\}| = 0$.
	\item Si $x_1\hdots x_n$ y $x_1\hdots y_n$ son palabras distintas, significa que deben diferir en al menos una letra; es decir, debe existir al menos un valor de $i$ tal que $x_i\neq y_i$, por lo que la cardinalidad de $\{i\in\{1,\hdots,n\}|x_i\neq y_i\}$ será como mínimo de $1$.
	\item Es inmediato que esto se cumple, ya que si $x_i\neq y_i$, entonces $y_i \neq x_i$.
	\item La desigualdad estricta se da cuando $\textbf{x}$ e $\textbf{y}$ tienen al menos una letra en común que es distinta en $\textbf{z}$. Si esto no ocurre, la igualdad se cumple. En cualquier caso, no puede haber una letra no coincida entre $\textbf{x}$ e $\textbf{y}$ y, sin embargo, coincida entre $\textbf{x}$ y $\textbf{z}$ y entre $\textbf{z}$ e $\textbf{y}$.
\end{enumerate}

$\qed$

\section{Ejercicio 6}

\begin{formulationBox}
	Sean $\textbf{x}, \textbf{y} \in \mathbb{F}_2^n$. Se define
	\[\textbf{x}\cap\textbf{y}=(x_1y_1, x_2y_2, \hdots, x_ny_n),\]
	donde $\textbf{x}=x_1\hdots x_n$ e $\textbf{y} = y_1\hdots y_n$. Demostrar que
	\[d(\textbf{x}, \textbf{y}) = w(\textbf{x}) + w(\textbf{y}) - 2w(\textbf{x}\cap\textbf{y}),\]
	donde $d(\textbf{x},\textbf{y})$ es la distancia de Hamming de las palabras $\textbf{x}$ e $\textbf{y}$ y $w(\textbf{z})$ es el peso de la palabra $\textbf{z}\in\mathbb{F}_2^n$.
\end{formulationBox}

Notemos que el alfabeto es $\mathbb{F}_2 = \{0, 1\}$. Por tanto, para que $x_iy_i = 1$, deben cumplirse tanto $x_i = 1$ como $y_i = 1$.

Es decir, que $w(\textbf{x}\cap\textbf{y})$ es el número de letras con valor $1$ que hay en común entre $\textbf{x}$ e $\textbf{y}$.

La distancia de Hamming es el número de letras que no tienen en común $\textbf{x}$ e $\textbf{y}$. Dada la naturaleza del alfabeto, esto es equivalente al número de letras con valor $1$ que no tienen en común $\textbf{x}$ e $\textbf{y}$.

Por tanto, podemos calcular a distancia de Hamming como el número de letras con valor $1$ que hay en total entre $\textbf{x}$ e $\textbf{y}$ (es decir, $w(\textbf{x}) + w(\textbf{y})$), descontando los que tienen en común dichas palabras (hay $w(\textbf{x}\cap\textbf{y})$ letras con valor $1$ en común; y se han contado por duplicado, una vez por cada palabra). Es decir, \[d(\textbf{x}, \textbf{y}) = w(\textbf{x}) + w(\textbf{y}) - 2w(\textbf{x}\cap\textbf{y}).\]

$\qed$

\section{Ejercicio 7}

\begin{formulationBox}
	Sea $C\subseteq T_n$ un código de longitud $n$ sobre el alfabeto $A = \{a_1,\hdots,a_m\}$ con distancia mínima $d$. Demostrar que si $\textbf{c}, \textbf{c'}\in C$ son dos palabras distintas de $C$, entonces $\overline{B}(\textbf{c}, \lfloor\frac{d-1}{2}\rfloor) \cap \overline{B}(\textbf{c'}, \lfloor\frac{d-1}{2}\rfloor) = \emptyset$.
\end{formulationBox}

Presentamos una prueba por contradicción.

Supongamos que existe una palabra $\textbf{p}$ tal que $\textbf{p}\in\overline{B}(\textbf{c}, \lfloor\frac{d-1}{2}\rfloor)$ y $\textbf{p}\in\overline{B}(\textbf{c'}, \lfloor\frac{d-1}{2}\rfloor)$. Es inmediato que tanto $\textbf{c}$ como $\textbf{c'}$ pertenecen a la bola $\overline{B}(\textbf{p}, \lfloor\frac{d-1}{2}\rfloor)$. El diámetro de la bola $\overline{B}(\textbf{p}, \lfloor\frac{d-1}{2}\rfloor)$ es $d-1$, luego la distancia entre $\textbf{c}$ y $\textbf{c'}$ es como mucho $d-1$, lo cual contradice la definición de $d$.

$\qed$

\section{Lema 3.2}

\begin{formulationBox}
	\begin{lemma}
		Sea $A = \{a_1,\hdots,a_m\}$ un alfabeto, $\textbf{x} = x_1\hdots x_n \in T_n$ y $0\leq r\leq n$. Entonces,
		\[|\overline{B}(\textbf{x}, r)| = \sum_{i=0}^r\binom{n}{i}(m-1)^i.\]
	\end{lemma}
\end{formulationBox}

\begin{proof}
	La expresión $|\overline{B}(\textbf{x}, r)|$ se refiere al número de palabras de $T_n$ que difieren de $\textbf{x}$ en $r$ letras o menos. Vamos a construir la fórmula para su cómputo.
	
	Llamamos \textit{esfera de centro $\textbf{x}$ y radio $r$} al conjunto
	\[S(\textbf{x}, r) = \{\textbf{y}\in A^n\mid\mathrm{d}(\textbf{x},\textbf{y}) = r\}.\]
	Podemos interpretar que las palabras en dicha esfera difieren de $\textbf{x}$ en exactamente $r$ letras.
	
	Puesto que la distancia de Hamming siempre devuelve números enteros, podemos reescribir la fórmula para el cardinal de la bola de la siguiente manera:
	\[|\overline{B}(\textbf{x}, r)| = \sum_{i=0}^r|S(\textbf{x}, i)|.\]
	
	Ahora la tarea se reduce a hallar una expresión para $|S(\textbf{x}, i)|$.
	
	Notemos que, si modificamos cualesquiera $i$ letras de $\textbf{x}$, podemos llegar a cualquier palabra en la esfera. Y, por definición de \textit{combinación}, tenemos exactamente $\binom{n}{i}$ posibles combinaciones de posiciones a modificar.
	
	Para cada una de las $i$ posiciones, podemos elegir entre $m-1$ letras diferentes a las que cambiar (ya que no podemos dejar ninguna de las posiciones sin modificar, debemos descontar la letra que viene en $\textbf{x}$). Es decir, que para cada combinación de posiciones tenemos $(m-1)^i$ modificaciones posibles.
	
	Por tanto,
	\[|S(\textbf{x}, i)| = \binom{n}{i}(m-1)^i,\]
	y de ahí sigue que
	\[|\overline{B}(\textbf{x}, r)| = \sum_{i=0}^r\binom{n}{i}(m-1)^i.\]
\end{proof}

\section{Ejercicio 8}

\begin{formulationBox}
	Sea $C\subseteq T_n$ un código de bloque de longitud $n$ sobre un alfabeto $A$ con distancia mínima $d$. Demostrar que $C$ es perfecto si y solo si $\dot{\bigcup}_{c\in C}\overline{B}(\textbf{c}, \lfloor\frac{d-1}{2}\rfloor) = T_n$.
\end{formulationBox}

Un código $C$ es perfecto si alcanza la cota de Hamming; es decir, si se cumple que
\[|C|\sum_{i=0}^{\lfloor\frac{d-1}{2}\rfloor} \binom{n}{i} (m-1)^i = m^n.\]

La expresión que hay en la parte izquierda hace referencia al número de palabras que están incluidas en alguna bola de radio $\lfloor\frac{d-1}{2}\rfloor$. Que el código alcance la cota de Hamming significa que el número de palabras que están incluidas en alguna bola con dicho radio es igual al número de palabras en $T_n$ (ya que $|T_n| = m^n$).

Notemos ahora que $\dot{\bigcup}_{c\in C}\overline{B}(\textbf{c}, \lfloor\frac{d-1}{2}\rfloor)$ es la unión de todas esas bolas. Evidentemente, $\dot{\bigcup}_{c\in C}\overline{B}(\textbf{c}, \lfloor\frac{d-1}{2}\rfloor) \subseteq T_n$. Si el código alcanza la cota de Hamming, $|\dot{\bigcup}_{c\in C}\overline{B}(\textbf{c}, \lfloor\frac{d-1}{2}\rfloor)| = m^n = |T_n|$. Como tenemos un subconjunto con la misma cardinalidad que su superconjunto, podemos concluir que son el mismo conjunto.

Similarmente, si sabemos que son el mismo conjunto, entonces sus cardinalidades deben ser iguales, por lo que el número de palabras incluidas en alguna bola será igual al número de palabras en total en $T_n$, que es precisamente equivalente a alcanzar la cota de Hamming.

$\qed$

\section{Ejercicio 9}

\begin{formulationBox}
	Demostrar que no existen códigos perfectos de longitud $n$ sobre un alfabeto $A$ con distancia mínima par.
\end{formulationBox}

Tomamos un código $C\subseteq T_n$ sobre $A$ tal que su distancia mínima $d$ sea par.

Notemos que existen palabras en $T_n$ que están a distancia $\frac{d}{2}$ a la palabra de $C$ más cercana. Esto es fácil de comprobar: si tomamos una palabra $\textbf{c}\in C$ y le cambiamos $\frac{d}{2}$ letras, debemos cambiar otras $\frac{d}{2}$ letras más para llegar a otra palabra $\textbf{c'}\in C$ tal que $\mathrm{d}(\textbf{c}, \textbf{c'})=d$, debido a la desigualdad triangular.

Ahora construimos el conjunto $\dot{\bigcup}_{c\in C}\overline{B}(\textbf{c}, \lfloor\frac{d-1}{2}\rfloor)$. Debido a que $d$ es par, el radio $\lfloor\frac{d-1}{2}\rfloor$ equivaldrá $\lfloor\frac{d-2}{2}+\frac{1}{2}\rfloor = \frac{d}{2}-1$. Es decir, las palabras a distancia $\frac{d}{2}$ no serán incluidas en el conjunto, luego $\dot{\bigcup}_{c\in C}\overline{B}(\textbf{c}, \lfloor\frac{d-1}{2}\rfloor) \neq T_n$ y, por el resultado obtenido en el ejercicio anterior, el código $C$ no será perfecto.
