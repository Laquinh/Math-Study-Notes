\chapter{Nociones básicas de la Teoría de Códigos}

\section{Ejercicio 1}

\begin{formulationBox}
	Estudiar si las siguientes tuplas corresponden a un código EAN:
	\begin{enumerate}[label=\alph*)]
		\item 9783540283713
		\item 8412345678914
		\item 9783662479735
	\end{enumerate}
\end{formulationBox}

Debemos estudiar si se satisface la siguiente ecuación:
\[3\sum_{i=0}^5a_{2i+1} + \sum_{i=0}^6a_{2i} \equiv 0\mod10.\]

\begin{enumerate}[label=\alph*)]
	\item
	\begin{align*}
		3\sum_{i=0}^5a_{2i+1} + \sum_{i=0}^6a_{2i} & = 3(7+3+4+2+3+1) + (9+8+5+0+8+7+3)\\
		&= 60 + 40\\
		&= 100 \equiv 0 \mod 10,
	\end{align*}
	por lo que $9783540283713$ es un código EAN.
	
	\item
	\begin{align*}
		3\sum_{i=0}^5a_{2i+1} + \sum_{i=0}^6a_{2i} & = 3(4+2+4+6+8+1) + (8+1+3+5+7+9+4)\\
		&= 75 + 37\\
		&= 112 \not\equiv 0 \mod 10,
	\end{align*}
	por lo que $8412345678914$ no es un código EAN.
	
	\item
	\begin{align*}
		3\sum_{i=0}^5a_{2i+1} + \sum_{i=0}^6a_{2i} & = 3(7+3+6+4+9+3) + (9+8+6+2+7+7+5)\\
		&= 96 + 64\\
		&= 110 \equiv 0 \mod 10,
	\end{align*}
	por lo que $9783662479735$ es un código EAN.
\end{enumerate}
