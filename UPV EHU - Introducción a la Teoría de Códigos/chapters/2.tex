\chapter{Nociones básicas de la Teoría de Códigos}

\section{Ejercicio 1}

\begin{formulationBox}
	Estudiar si las siguientes tuplas corresponden a un código EAN:
	\begin{enumerate}[label=\alph*)]
		\item 9783540283713
		\item 8412345678914
		\item 9783662479735
	\end{enumerate}
\end{formulationBox}

Debemos estudiar si se satisface la siguiente ecuación:
\[3\sum_{i=0}^5a_{2i+1} + \sum_{i=0}^6a_{2i} \equiv 0\mod10.\]

\begin{enumerate}[label=\alph*)]
	\item
	\begin{align*}
		3\sum_{i=0}^5a_{2i+1} + \sum_{i=0}^6a_{2i} & = 3(7+3+4+2+3+1) + (9+8+5+0+8+7+3)\\
		&= 60 + 40\\
		&= 100 \equiv 0 \mod 10,
	\end{align*}
	por lo que $9783540283713$ es un código EAN.
	
	\item
	\begin{align*}
		3\sum_{i=0}^5a_{2i+1} + \sum_{i=0}^6a_{2i} & = 3(4+2+4+6+8+1) + (8+1+3+5+7+9+4)\\
		&= 75 + 37\\
		&= 112 \not\equiv 0 \mod 10,
	\end{align*}
	por lo que $8412345678914$ no es un código EAN.
	
	\item
	\begin{align*}
		3\sum_{i=0}^5a_{2i+1} + \sum_{i=0}^6a_{2i} & = 3(7+3+6+4+9+3) + (9+8+6+2+7+7+5)\\
		&= 96 + 64\\
		&= 110 \equiv 0 \mod 10,
	\end{align*}
	por lo que $9783662479735$ es un código EAN.
\end{enumerate}

\section{Ejercicio 2}

\begin{formulationBox}
	Determinar el valor de $a$ para que las siguientes tuplas correspondan a un código EAN:
	\begin{enumerate}[label=\alph*)]
		\item 843a554161836
		\item 4325351455a52
		\item 978421345667a
	\end{enumerate}
\end{formulationBox}

\begin{enumerate}[label=\alph*)]
	\item
	\begin{align*}
		3\sum_{i=0}^5a_{2i+1} + \sum_{i=0}^6a_{2i} & = 3(4+a+5+1+1+3) + (8+3+5+4+6+8+6)\\
		&= 3a + 42 + 40\\
		&= 3a + 82 \equiv 0 \mod 10 \implies 3a \equiv 8 \mod 10 \implies a = 6.
	\end{align*}
	resultando en el código EAN $8436554161836$.
	
	\item
	\begin{align*}
		3\sum_{i=0}^5a_{2i+1} + \sum_{i=0}^6a_{2i} & = 3(3+5+5+4+5+5) + (4+2+3+1+5+a+2)\\
		&= 81 + 17 + a\\
		&= 98 + a \equiv 0 \mod 10 \implies a \equiv 2 \mod 10 \implies a = 2,
	\end{align*}
	resultando en el código EAN $4325351455252$.
	
	\item
	\begin{align*}
		3\sum_{i=0}^5a_{2i+1} + \sum_{i=0}^6a_{2i} & = 3(7+4+1+4+6+7) + (9+8+2+3+5+6+a)\\
		&= 87 + 33 + a\\
		&= 120 + a \equiv 0 \mod 10 \implies a \equiv 0 \mod 10 \implies a = 0,
	\end{align*}
	resultando en el código EAN $9784213456670$.
\end{enumerate}

\section{Ejercicio 3}

\begin{formulationBox}
	Determinar la distancia mínima del código EAN. Deducir cuántos errores detecta y corrige. Probar que el código detecta la transposición de dos dígitos consecutivos; es decir, del cambio de la palabra $a_0\dots a_ia_{i+1}a_{12}$ del código por $a_0\dots a_{i+1}a_i\dots a_{12}$, con $i\in\{0,\dots,11\}$.
\end{formulationBox}

Si tomamos una palabra del código EAN y le cambiamos un solo dígito, siempre nos veremos obligados a cambiar uno más para compensarlo y que vuelva a satisfacerse la ecuación. Por lo tanto, no hay palabras en el código que difieran entre sí en un solo dígito, mas sí en $2$, por lo que la distancia mínima del código EAN es $d(C_{EAN})=2$.

Siguiendo el mismo razonamiento, si tomamos una palabra $\textbf{c}\in C_{EAN}$ y le cambiamos un solo dígito, obteniendo una nueva palabra $\textbf{y}$, la ecuación no va a satisfacerse para $\textbf{y}$, por lo que podremos detectar el error. Esto no ocurre cuando hay más de un cambio. Por tanto, el código EAN detecta hasta 1 error.

Sin embargo, si recibimos la palabra $\textbf{y}$, hay múltiples palabras $\textbf{c}\in C_{EAN}$ tales que $d(\textbf{c}, \textbf{y}) = 1$; es decir, la palabra $\textbf{y}$ no tiene una decodificación única, luego el código EAN no corrige ningún error.

Al contrario de lo que insinúa el enunciado, el código EAN no siempre es capaz de detectar la transposición de dos dígitos consecutivos. Por ejemplo, si tomamos la palabra $\textbf{c} = 8771493526675$ y le aplicamos la transposición a los dígitos consecutivos $4$ y $9$, obtenemos la palabra $\textbf{c}' = 8771943526675$, que ¡también pertenece a $C_{EAN}$! y, por lo tanto, no va a ser detectado el error.
