\documentclass[11pt, a4paper, tikz]{article}

\usepackage[english]{babel} %language
\usepackage[utf8]{inputenc} %UTF-8 encoding
\usepackage[margin=1in]{geometry} %smaller margin
\usepackage{amssymb} %\nexists \mathbb
\usepackage{amsmath}
\usepackage{amsthm}
\usepackage{graphicx}
\usepackage{polynom}
\usepackage{mathtools}
\usepackage{enumitem}

\usepackage{xlop}
\usepackage[dvipsnames]{xcolor}
\usepackage{tcolorbox}

\definecolor{formulationFrameColor}{RGB}{255,255,255}
\definecolor{formulationBackgroundColor}{RGB}{240,245,250}

\newtcolorbox{formulationBox}{
	colframe=formulationFrameColor,
	colback=formulationBackgroundColor
}

\usepackage{tikz}   
\usepackage{pgfplots}

\pgfplotsset{compat=1.6}

\pgfplotsset{soldot/.style={color=red,only marks,mark=*}} \pgfplotsset{holdot/.style={color=red,fill=white,only marks,mark=*}}

\pgfplotsset{compat=newest} % used to declare my style command.  

\newcommand{\newpara}{
	\vskip 2mm
}

\newcommand{\centsection}[1]{
	\section*{\centering{#1}}
}

\newcommand{\centsubsection}[1]{
	\subsection*{\centering{#1}}
}

\newcommand{\myOver}[2]{
	\ensuremath{\overset{\kern2pt #1}{#2}}
}

\newcommand{\final}[1]{
	$\mathcal{F}($#1$)$
}

\renewcommand{\qed}{\hfill\blacksquare}

\newcommand{\Lim}[1]{\raisebox{0.5ex}{\scalebox{0.8}{$\displaystyle \lim_{#1}\;$}}}
\newcommand{\Inf}[1]{\raisebox{0.5ex}{\scalebox{0.8}{$\displaystyle \inf_{#1}\;$}}}
\newcommand{\Sup}[1]{\raisebox{0.5ex}{\scalebox{0.8}{$\displaystyle \sup_{#1}\;$}}}
\newcommand{\Sum}[2]{\displaystyle \sum_{#1}^{#2}}
\newcommand{\Int}[2]{\displaystyle \int_{#1}^{#2}}
\newcommand{\Bigcup}[2]{\displaystyle \bigcup_{#1}^{#2}}

\newcommand{\naturals}{
	\ensuremath{\mathbb{N}}
}
\newcommand{\integers}{
	\ensuremath{\mathbb{Z}}
}
\newcommand{\rationals}{
	\ensuremath{\mathbb{Q}}
}
\newcommand{\reals}{
	\ensuremath{\mathbb{R}}
}
\newcommand{\complexes}{
	\ensuremath{\mathbb{C}}
}

\newcommand{\cover}[1]{
	\ensuremath{\mathcal{#1}}
}

\newcommand{\sigmaAlgebra}[1]{
	\ensuremath{\mathcal{#1}}
}

\graphicspath{ {./media/} }

\begin{document}
	\title{\textbf{Chapter 2 — Section B}}
	\maketitle
	%\setcounter{section}{3}
	\centsection{Exercise 1}
	
	\begin{formulationBox}
		Show that $\sigmaAlgebra{S} = \{\bigcup_{n\in K}(n, n+1]:K\subseteq\integers\}$ is a $\sigma$-algebra on $\reals$.
	\end{formulationBox}

	\begin{itemize}
	\item First off, we prove the first bullet point. Since $\varnothing\subseteq\integers$, an element of $\sigmaAlgebra{S}$ is $\bigcup_{n\in\varnothing}(n, n+1] = \varnothing$.
	
	\item As for the second bullet point, we note that $(a, a+1]\cap(b, b+1] = \varnothing$ for all $a,b\in\integers$ with $a\neq b$. This may be understood as each interval $(n, n+1]$ being \textit{generated by} a unique number $n\in\integers$.
	
	Now, let $E = \bigcup_{n\in K}(n, n+1]$ for some random $K\subseteq\integers$. The element $\bar{E} = \reals\setminus E$ can be defined as $\bar{E} = \bigcup_{n\in \bar{K}}(n, n+1]$, where $\bar{K} = \integers\setminus K$. Since $\bar{K}\in\integers$, we get $\bar{E}\in\sigmaAlgebra{S}$.
	
	\item Finally, we prove the third bullet point. Let $E_1, E_2, ...$ be a sequence of elements of $\sigmaAlgebra{S}$. Of course, each $E_k$ is the union of some intervals $(n, n+1]$, with $n\in K_k$ for some $K_k\subseteq\integers$. This may be understood as each element $E_k$ being \textit{generated by} a unique subset $K_k\subseteq\integers$.
	
	Since the union is asociative, for some random $E_a$, $E_b$ in the sequence, we get $E_a\cup E_b = \bigcup_{n\in(K_a\cup K_b)}(n, n+1]$. Note that $K_a\cup K_b$ contains generators from both $K_a$ and $K_b$. In general, $\Bigcup{k=1}{\infty}E_k = \bigcup_{n\in\bigcup_{k=1}^{\infty}K}(n, n+1]$. Since $\bigcup_{k=1}^{\infty}K \subseteq \integers$, we conclude that $\bigcup_{k=1}^{\infty}E_k \in \sigmaAlgebra{S}$.
	
	\end{itemize}
	
	$\qed$
\end{document}