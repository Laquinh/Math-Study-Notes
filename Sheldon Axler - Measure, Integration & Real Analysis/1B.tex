\documentclass[11pt, a4paper, tikz]{article}

\usepackage[english]{babel} %language
\usepackage[utf8]{inputenc} %UTF-8 encoding
\usepackage[margin=0.5in]{geometry} %smaller margin
\usepackage{amssymb} %\nexists \mathbb
\usepackage{amsmath}
\usepackage{amsthm}
\usepackage{graphicx}
\usepackage{polynom}
\usepackage{mathtools}

\usepackage{xlop}
\usepackage[dvipsnames]{xcolor}
\usepackage{tcolorbox}

\definecolor{formulationFrameColor}{RGB}{150,160,190}
\definecolor{formulationBackgroundColor}{RGB}{240,245,250}

\newtcolorbox{formulationBox}{
	colframe=formulationFrameColor,
	colback=formulationBackgroundColor
}

\usepackage{tikz}   
\usepackage{pgfplots}

\pgfplotsset{compat=1.6}

\pgfplotsset{soldot/.style={color=red,only marks,mark=*}} \pgfplotsset{holdot/.style={color=red,fill=white,only marks,mark=*}}

\pgfplotsset{compat=newest} % used to declare my style command.  

\newcommand{\newpara}{
	\vskip 2mm
}

\newcommand{\centsection}[1]{
	\section*{\centering{#1}}
}

\newcommand{\centsubsection}[1]{
	\subsection*{\centering{#1}}
}

\newcommand{\myOver}[2]{
	\ensuremath{\overset{\kern2pt #1}{#2}}
}

\newcommand{\final}[1]{
	$\mathcal{F}($#1$)$
}

\renewcommand{\qed}{\hfill\blacksquare}

\newcommand{\Lim}[1]{\raisebox{0.5ex}{\scalebox{0.8}{$\displaystyle \lim_{#1}\;$}}}
\newcommand{\Inf}[1]{\raisebox{0.5ex}{\scalebox{0.8}{$\displaystyle \inf_{#1}\;$}}}
\newcommand{\Sup}[1]{\raisebox{0.5ex}{\scalebox{0.8}{$\displaystyle \sup_{#1}\;$}}}
\newcommand{\Sum}[2]{\displaystyle \sum_{#1}^{#2}}
\newcommand{\Int}[2]{\displaystyle \int_{#1}^{#2}}

\newcommand{\naturals}{
	\ensuremath{\mathbb{N}}
}
\newcommand{\integers}{
	\ensuremath{\mathbb{Z}}
}
\newcommand{\rationals}{
	\ensuremath{\mathbb{Q}}
}
\newcommand{\reals}{
	\ensuremath{\mathbb{R}}
}
\newcommand{\complexes}{
	\ensuremath{\mathbb{C}}
}

\begin{document}
	\title{\textbf{Chapter 1 — Section B}}
	\maketitle
	%\setcounter{section}{3}
	\centsection{Exercise 1}
	
	\begin{formulationBox}
		Define $f:[0,1]\rightarrow\reals$ as follows:
		\[
			f(a) =
			\begin{cases}
				0 &\quad \textrm{if $a$ is irrational,}\\
				\frac{1}{n} &\quad \parbox{.5\linewidth}{if $a$ is rational and $n$ is the smallest positive integer such that $a=\frac{m}{n}$ for some integer $m$.}
			\end{cases}
		\]
		Show that $f$ is Riemann integrable and compute $\Int{0}{1}f$.
	\end{formulationBox}
	
	Let $P_k$ denote partitions of $[0,1]$.
	
	Every $P_k$ contains irrational numbers, which implies that the infimum of the function in each subinterval of $P_k$ is 0, thus $L(f, P_k, [0,1]) = 0\ \forall P_k$ and, therefore, $L(f, [0,1]) = 0$.
	
	At the same time, every $P_k$ contains rational numbers.
	
	Let $\mathcal{R} = r_1, r_2, \dots$ denote the rational numbers in $[0, 1]$. These can be arranged in a way such that, for any $\epsilon>0$, there exists some $n\in\naturals$ such that $f(r_k)<\epsilon$ for all $k\geq n$.
	
	We define $g:[0,1]\rightarrow\reals$ as $g(a) = 0$. Trivially, $g(a)=f(a)$ except when $a\in\mathcal{R}$, where $g(a)-f(a) = \frac{1}{n}$ by given in the definition above.
	
	%As there are a finite number of rationals until $r_k$, by the theorem proven in exercise 6, we know that these are not going to affect the value of the integral. On the other hand, while 
	
	%use the fact that $\frac{1}{n^2}$ is divergent (?)
	
	\centsection{Exercise 4}
	
	\begin{formulationBox}
		Give an example of bounded functions $f, g:[0,1]\rightarrow\reals$ such that \[L(f, [0,1]) + L(g, [0,1]) < L(f+g,[0,1])\] and \[U(f+g,[0,1]) < U(f, [0,1]) + U(g, [0,1]).\]
	\end{formulationBox}
	
	Define $f:[0,1]\rightarrow\reals$ as
	\[
		f(x) =
		\begin{cases}
			0 &\quad \textrm{if $x$ is irrational,}\\
			1 &\quad \textrm{if $x$ is rational}
		\end{cases}
	\]
	
	and $g:[0,1]\rightarrow\reals$ as
	\[
		g(x) =
		\begin{cases}
			1 &\quad \textrm{if $x$ is irrational,}\\
			0 &\quad \textrm{if $x$ is rational.}
		\end{cases}
	\]
	
	Then, $(f+g)(x) = 1\ \forall x\in[0,1]$. The first inequality holds: $0 + 0 < 1$, and the second one as well: $1 < 1 + 1$.
	
	$\qed$
\end{document}